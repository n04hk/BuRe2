\section{Prüfung 13.08.2014}
\subsection{Immaterialgüterrecht}
Rechts- oder Schutzgut Ja/Nein?\\
\begin{tabular}{|l|l|}
	\hline 
	\textbf{Ja} & \textbf{Nein} \\ 
	\hline 
	Name 'Prada' auf Schuh & Schuldbrief \\ 
	\hline 
	Aufsatz von Person, der in Schule geschrieben wurde & Eine auf swissreg veröffentlichte Patentschrift \\
	\hline
	Qualitätshinweis 'IP-Swiss' auf Verpackung & Von Mozart (1791) komponierte Senate \\
	\hline
	vegetative Rosenzüchtung namens 'rote Elise' &  \\
	\hline
	Bezeichnung 'smart7' für Zahnpasta &  \\
	\hline
	Domain-Name 'nestle.ch' &  \\
	\hline
	Dissertation &  \\
	\hline
\end{tabular}

\subsection{Urheberrecht}
\begin{itemize}
	\item Mit dem \textbf{Privatkonkurs} des Urhebers geht sein \textbf{Urheberpersönlichkeitsrecht} \textbf{nicht} unter.
	\item Eine gekaufte DVD darf \textbf{nicht} zum doppelten Preis übers \textbf{Internet} verkauft werden.
	\item Der Inhaber eines \textbf{Urhebernutzungsrecht} ist \textbf{nicht} stets der Urheber.
	\item Das Kopieren eines PC-Programms ist auch zu \textbf{Unterrichtszwecken} \textbf{nicht} erlaubt.
	\item Das Kopieren von Auszügen aus einem Fachbuch zu \textbf{Unterrichtszwecken} ist erlaubt.
	\item Die rechtmässige Werkverwendung zu \textbf{Unterrichtszwecken} ist \textbf{entgeltlich}.
	\item \textbf{Domain-Namen} können die Rechte an einer \textbf{Bildmarke} \textbf{nicht} verletzen.
	\item Der \textbf{Switch-Domain-Namen-Streitbeilegungsdienst} ist \textbf{nicht} ein halbstaatliches Gericht.
	\item Die rechtmässige Werkverwendung innerhalb des \textbf{privaten Kreises} ist unmittelbar \textbf{unentgeltlich}.
	\item \textbf{Domain-Namen} sind \textbf{nicht} Werke im Sinne der Urheberrechts.
	\item Das CH-Urheberrecht erlaubt das beliebige \textbf{Kopieren} von \textbf{Computerprogrammen} zu Privatzwecken \textbf{nicht}.
	\item Der Erwerber einer \textbf{CD} darf diese einem beliebigen Dritten \textbf{vermieten}.
	\item Der Erwerber eines \textbf{Downloads} eines \textbf{PC-Programms} darf dieses von Gesetzes wegen einem Dritten ohne Zustimmung des Rechtsinhabers \textbf{nicht} verschenken.
	\item Eine in \textbf{DE} eingetragene Marke zeitigt \textbf{nicht} gleichzeitig \textbf{Schutzwirkungen} auf die CH.
	\item Das \textbf{Urheberpersönlichkeitsrecht} an einem in Erfüllung des Arbeitsvertrags geschaffenen \textbf{PC-Programms} geht \textbf{nicht} auf den Arbeitgeber über.
	\item Das Recht zum \textbf{Reverse-Engineering} von Schnittstelleninformationen bedeutet \textbf{nicht} eine Ausweitung des \textbf{Rechtschutzes} zu Gunsten des Rechteinhabers.
	\item Das Recht zum \textbf{Reverse-Engineering} von Schnittstelleninformationen kann \textbf{nicht} ausschliesslich nur von den \textbf{Verwertungsgesellschaften} wahrgenommen werden.
	\item Der Eigentümer eines \textbf{Werkexemplars} ist \textbf{nicht} gleichzeitig auch der Inhaber der Urheberrechte am im \textbf{Werkexemplar} ausgedrückten Wert.
	\item Das Marken-/Firmenrecht sind dem \textbf{Kennzeichenrecht} zuzuordnen.
	\item Das Recht zum \textbf{Reverse-Engineering} von Schnittstelleninformationen bedeutet \textbf{nicht} eine Beschränkung des \textbf{Urheberpersönlichkeitsrechts} des Urhebers des PC-Programms.
	\item Gemäss CH-Rechtsauffassung sind \textbf{Urheberrechte dinglich übertragbar}.
	\item Die Rechte an einem \textbf{Benutzerhandbuch} bestimmen sich \textbf{nicht} nach Art. 2 Abs. 3 URG.
	\item Eine neue Marke, welche mit einer älteren Marke identisch und für die gleichen Waren oder Dienstleistungen bestimmt ist, wird durch das \textbf{IGE} ins \textbf{Markenregister} eingetragen.
\end{itemize} 

\subsection{AGB: Datenbearbeitungserklärung}
Welche Punkte muss diese beinhalten:
\begin{itemize}
	\item Dass die von einem Interessenten im Rahmen der Erstellung einer konkreten Offerte erhobenen Daten von der Firma dazu verwendet werden dürfen, dem betreffenden Interessenten anschliessend per \textbf{E-Mail Angebote} zuzustellen.
	\item Dass die IP-Adresse des Benutzers der Internetseite der Firma sowie das Datum und die Uhrzeit des Zugriffs durch denselben an einen Server von Google Inc. in den \textbf{USA} übertragen, dort gespeichert und zu statistischen Zwecken ausgewertet werden.
\end{itemize}

\subsection{AGB: Datenbearbeitungserklärung}
AGB ohne Datenbearbeitungserklärung, wann dürfen Personendaten bearbeitet werden?
\begin{itemize}
	\item In unmittelbarem Zusammenhang mit dem Abschluss eines Vertrags über den Verkauf eines Gebrauchtwagens darf die Firma die dafür benötigten \textbf{Personendaten} des Käufers auch ohne explizite Ermächtigung in den AGB \textbf{bearbeiten}.
	\item Die \textbf{Kundekorrespondenz und Rechnungsbelege} darf die Firma auch ohne dass dies in den AGB erwähnt wäre, während \textbf{10 Jahren} aufbewahren.
\end{itemize}

\subsection{AGB}
Teil der AGB: Bei IT-Autoherstellern haftet die Firma nicht für Schäden aufgrund von arglistig verschwiegenen Mängeln. Wenn ein auf Internetseite angebotenes Auto bereits verkauft ist, darf die Firma ein qualitativ und preislich gleichwertiges Fahrzeug aushändigen. Der Käufer ist verpflichtet, sein Auto regelmässig und ausschliesslich bei dieser Firma warten und reparieren zu lassen. Ausschliesslicher Gerichtsstand für sämtliche Streitigkeiten aus Verträgen zwischen der Firma und dem Kunden ist Rapperswil-Jona SG.\\
Welche Aussagen bezüglich AGB-Bestimmungen sind richtig?
\begin{itemize}
	\item Der Haftungsausschluss für Schäden aufgrund von arglistig verschwiegenen Mängeln bei Fahrzeugen von IT-Herstellern ist \textbf{widerrechtlich} und deshalb \textbf{nichtig}.
	\item Der Käufer \textbf{muss nicht damit rechnen}, dass die Firma berechtigt ist, ihm anstatt des auf ihrer Internetseite angebotenen Fahrzeugs ein qualitativ und preislich gleichwertiges Fahrzeug auszuhändigen, weshalb die	betreffende AGB Klausel \textbf{keine Geltung} erlangt.
	\item Der \textbf{ausschliessliche Gerichtsstand} in Rapperswil-Jona/SG ist zumindest für Verträge, die ein Kunde als Konsument mit der Firma abschliesst, \textbf{ungültig}.
\end{itemize}

\subsection{AGB: Widerruf}
Die AGB der Firma enthalten keine Aussage dazu, ob ein Käufer die Möglichkeit hat, sich von einem bereits abgeschlossenen Kaufvertrag durch Erklärung eines Widerrufs zu lösen. Auch ausserhalb der AGB weist die Firma ihre Kunden nicht auf eine solche Möglichkeit hin. Kann ein Kunde mit Wohnsitz in der CH auch ohne einen expliziten Hinweis durch die Firma einen mit dieser abgeschlossenen Vertrag nachträglich widerrufen?
\begin{itemize}
	\item Nein, denn ein Käufer mit Wohnsitz in der \textbf{CH} besitzt gemäss heutiger Rechtslage gegenüber einem Verkäufer wie der Firma grundsätzlich nur ein \textbf{Widerrufsrecht}, wenn der Verkäufer ihm dieses Recht \textbf{freiwillig einräumt}.
\end{itemize}

\subsection{Straftatbestände}
Michael G. arbeitete während mehreren Jahren als IT-Mitarbeiter bei der Firma, bevor er dort Ende November 2011 fristlos entlassen wurde, weil er wiederholt unentschuldigt nicht zur Arbeit erscheinen war. Nachdem er seinen Arbeitsplatz geräumt hatte, nutzte Michael G. einen unbeobachteten Augenblick und erstellte in der Benutzerverwaltung des internen Datenverarbeitungssystems noch schnell ein neues Benutzerkonto und stattete dieses mit Rechten eines Domänen-Administrators aus. So konnte Michael G. auch nach Beendigung seines Arbeitsverhältnisses unbemerkt von extern auf den gegen unbefugten Zugriff besonders gesicherten Server der Firma zugreifen. Bis Ende Mai 2013, also während mehr als eineinhalb Jahren, verlustierte Michael G. sich regelmässig im Outlook-Kalender seiner Ex-Arbeitskollegen und löschte oder verschob Kundentermine zeitlich, so dass sie nicht eingehalten werden konnten. Dies führte dazu, dass der Firma mehrere neue Verkaufsabschlüsse
entgingen.\\
Hat G sich strafbar gemacht und falls ja, inwiefern?
\begin{itemize}
	\item \textbf{Unbefugtes Eindringen in ein Datenverarbeitungssystem} (Art. 143bis StGB)
	\item \textbf{Datenbeschädigung} (Art. 144bis StGB)
\end{itemize}