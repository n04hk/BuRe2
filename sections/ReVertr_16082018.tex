\section{Prüfung 16.08.2018}
\subsection{Kennzeichen/Wortmarke}
Die All Security AG bewirbt seit ihrer \textbf{Gründung} am 15. Mai 2017 ihre Dienstleistungen im Sicherheitsbereich, nämlich die Bewachung von Objekten und Personen, unter dem \textbf{Kennzeichen} ''Security''. Ihre Mitbewerberin, die Security for all AG, ist seit dem 1. Februar 2018 Inhaber der \textbf{Schweizer Wortmarke} MEIER Security (\textbf{Hinterlegung} am 15. Oktober 2015) mit Schutzbereich ''Sicherheitsdienstleistungen'' der Waren- und Dienstleistungsklasse 45 und ''Versicherungsdienstleistungen'' der Waren- und Dienstleistungsklasse 35.\\
Sind die nachstehenden Behauptungen richtig?
\begin{itemize}
	\item Die Security for all AG darf der All Security AG die Benutzung des \textbf{Kennzeichens} ''Security'' für die beanspruchte Waren- und Dienstleistungsklasse 45 \textbf{nicht} verbieten.
	\item All Security AG darf das \textbf{Kennzeichen} ''Security'' für das von ihr betriebene Tanzlokal in Rapperswil-Jona verwenden.
	\item All Security AG darf das \textbf{Kennzeichen} ''Security'' für das von ihr angebotene Versicherungsprodukt für Vandalenschäden an Gebäuden verwenden.
\end{itemize}

\subsection{Immaterialgüterrecht}
\begin{itemize}
	\item Das Immaterialgüterrecht kann man \textbf{nicht} als \textbf{körperliches Eigentum} bezeichnen, da das Wissen durch ''körperliche Arbeit'' erstellt wurde.
	\item Warum ist die \textbf{Geheimhaltung} von \textbf{Technologie} wichtig für eine \textbf{Patentanmeldung}? Man kann ein Patent \textbf{nicht} nur anmelden, wenn die Technologie vor dem Anmeldetag veröffentlicht wurde.
	\item Die \textbf{Patentansprüche} müssen den Gegenstand angeben für den \textbf{Schutz} begehrt wird und sind der \textbf{wichtigste Teil einer Patentschrift}.
	\item Eine \textbf{Erfindung} kann zum \textbf{Patent} angemeldet werden, wenn der Gegenstand der Erfindung \textbf{neu} ist, wenn der Gegenstand sich für den \textbf{Fachmann} nicht in nahe liegender Weise aus dem \textbf{Stand der Technik} ergibt und wenn die Erfindung \textbf{gewerblich anwendbar} ist.
	\item PVÜ (\textbf{Pariser Verbandsübereinkunft}) ist die Vorstufe zum \textbf{Europäischen Patentübereinkommen}.
	\item Das \textbf{TRIPS} (Agreement on trade-related aspects of Intellectual Property Rights) bezieht sich auf die \textbf{Vermarktung} und den \textbf{Handel} von Rechten am \textbf{geistigen Eigentum}.
	\item Was für wichtige Informationen kann man einer \textbf{Patentanmeldung ohne Recherchenbericht} entnehmen?
	\begin{itemize}
		\item Anmeldetag
		\item Anmelder
		\item Vertragsstaaten
		\item Zeichnungen
		\item Patentansprüche
	\end{itemize}
	\item \textbf{Software} ist nach dem \textbf{Europäischen Patentübereinkommen} patentierbar, in dem eine Erfindung ein \textbf{PC-Programm enthalten darf}.
\end{itemize}

\subsection{Urheberrecht}
\begin{itemize}
	\item Das Urheber\textbf{persönlichkeits}recht ist \textbf{nicht} vererbbar.
	\item Eine im \textbf{Handelsregister} eingetragene Firma, verliert ihren \textbf{Rechtsschutz} bis zur \textbf{Löschung} nicht.
	\item \textbf{Domain-Namen} können die Rechte an einer \textbf{Wortmarke} verletzen.
	\item Die rechtmässige \textbf{Werkverwendung} für \textbf{Unterrichtszwecke} (Art. 19 Abs. 1 lit. b URG) ist \textbf{nicht} unentgeltlich.
	\item \textbf{Datenbanken} sind in der \textbf{EU} \textbf{nicht} ausschliesslich durch das \textbf{Lauterkeitsrecht (Wettbewerbsrecht)} geschützt.
\end{itemize}

\subsection{Umgang mit Kundendaten: Nachteil für Kunden}
Firma erfasst bei Bestellung Vor- und Nachname, Adresse, Geburtsdatum etc. Diese Daten möchte sie für zielgerichtete Werbung und individuelle Rabattaktionen verwenden. Kunden, die die Webseite über einen PC, Smartphone, Tablet eines bestimmten Premiumherstellers anwählen für dieselben Produkte einen höheren Preis angezeigt, als Benutzer anderer Hersteller von IT-Hardware. Bei Bestellung eines Produkts erhält der Kunde automatische einen Newsletter zugesendet. Ohne den Kunden darauf hinzuweisen, verwendet die Firma zudem Cookies. Schliesslich speichert die Firma sämtliche Kundendaten, welche sie zu Marketingzwecken bearbeitet, in einer Cloud auf einem externen Server in den USA.\\
Welche Nachteile können diese Massnahmen für die Kunden der Firma haben?
\begin{itemize}
	\item Das ''\textbf{Personal Pricing}'' der Firma bedeutet, dass einem Kunden aufgrund seiner (vermeintlich höheren) Kaufkraft wegen Verwendung bestimmter IT-Hardware \textbf{höhere Preise} angezeigt werden können, womit dieser bei einem Kauf für dasselbe Produkte gegebenenfalls
	mehr bezahlen würde, als ein anderer Kunde.
	\item Falls der Kunde die Zustellung eines \textbf{Newsletters} über neue Modeartikel durch die Firma nicht abwählen kann, erfolgt die Zusendung von Werbung und die damit verbundene Bearbeitung seiner Personendaten \textbf{nicht wirklich freiwillig}.
	\item Die \textbf{Bearbeitung von Personendaten} von Kunden mittels Cookies ohne einen entsprechenden Hinweis durch die Firma bedeutet, dass der Kunde über die Datenbearbeitung nicht informiert wird und diese folglich \textbf{ohne dessen Zustimmung} erfolgt.
	\item Die \textbf{Speicherung von Personendaten} der Kunden der Firma auf einem externen Server in den \textbf{USA} bedeutet für diese einen Nachteil, da die USA gegenüber der Schweiz \textbf{keinen} gleichwertigen gesetzlichen Datenschutz gewährleisten.
\end{itemize}

\subsection{Umgang mit Kundendaten}
Die Firma aus vorheriger Aufgabe möchte die Personendaten ihrer Kunden besser schützen. Sie möchte technische und organisatorische Massnahmen treffen, um die Integrität, Verfügbarkeit und Vertraulichkeit der bearbeiteten Daten sicherzustellen.\\
Was bedeuten diese Massnahmen konkret?
\begin{itemize}
	\item Die \textbf{Integrität der Daten} ist gewährleistet, wenn die Firma sicherstellt, dass die Daten ihrer Kunden korrekt, vollständig und aktuell sind.
	\item Die \textbf{Verfügbarkeit} stellt sicher, dass Daten zur Verfügung stehen, wenn sie gebraucht werden, bspw. wenn ein Kunde ein \textbf{Auskunftsgesuch} bei der Firm stellt.
	\item Die \textbf{Vertraulichkeit} bedeutet, dass die Firma zu \textbf{verhindern} hat, dass die Daten ihrer Kunden von \textbf{Unbefugten} zur Kenntnis genommen werden könnte.
	\item Die \textbf{Vertraulichkeit} bedeutet \textbf{nicht}, dass die Firma die Daten ihrer Kunden nur zu jenem Zweck zu bearbeiten hat, welcher bei der Beschaffung derselben von ihr angegeben wurde.
\end{itemize}

\subsection{AGB}
Firma aus vorheriger Aufgabe, AGB: Die Angebote im Online-Shop der Toscount AG sind freibleibend und unverbindlich. Bestellungen sind für die Toscount AG erst nach schriftlicher Bestätigung durch die Toscount AG verbindlich. Eine Rücksendung von Artikeln durch den Kunden bedarf der vorherigen Zustimmung der Toscount AG und erfolgt auf Kosten und Risiko des Kunden. Die Rechnungen sind je nach Vereinbarung per Nachnahme, bar oder innert 10 Tagen rein netto zahlbar, soweit nicht anders vereinbart. Bezahlt der Kunde die Rechnung nicht innert der Zahlungsfrist, so schuldet er der Toscount AG eine pauschale Umtriebsgebühr von Fr. 500.–, welche automatisch dem Kundenkonto belastet oder dem Kunden in Rechnung gestellt wird. Lieferungen erfolgen nur innerhalb der Schweiz an die Adresse, die bei der Registrierung oder bei der Bestellung angegeben wird. Mägenwil ist ausschliesslich Gerichtsstand für alle sich aus dem Vertragsverhältnis unmittelbar oder mittelbar ergebenden Streitigkeiten. Das Rechtsverhältnis untersteht dem schweizerischen Recht.
\begin{itemize}
	\item Die Aussage, wonach die \textbf{Bestellung} eines Kunden erst nach \textbf{schriftlicher Bestätigung} durch die Toscount AG verbindlich ist, lässt darauf schliessen, dass die Angebote auf der Toscount-Website unverbindlich sind und der Kunde bei einer Online-Bestellung den \textbf{Antrag zum Abschluss eines Kaufvertrages} stellt.
	\item Die pauschale \textbf{Umtriebsgebühr} von Fr. 500.– gilt nicht, da der Kunde \textbf{nicht} mit einer solchen Bestimmung in den AGB eines Online-Shops wie der Toscount AG rechnen muss.
	\item Die Regelung, wonach die \textbf{Rücksendung von Produkten} durch den Kunden der \textbf{vorherigen Zustimmung} der Toscount AG bedarf, ist gültig, da der Käufer von Modeartikeln über einen schweizerischen Online-Shop in der Schweiz \textbf{nicht} automatisch über ein 14-tägiges \textbf{Widerrufsrecht} verfügt.
	\item Der \textbf{ausschliessliche Gerichtsstand} in Mägenwil gilt \textbf{nicht} für eine durch die Toscount AG erhobene gerichtliche Klage gegen einen Kunden mit \textbf{Wohnsitz in der Schweiz} aus einem Kaufvertrag für Modeartikel, die der Kunde für seinen persönlichen oder familiären Bedürfnisse bei der Toscount AG bestellt hat.
\end{itemize}

\subsection{AGB}
Die Firma aus vorheriger Aufgabe möchte nun auch nach Italien liefern. Sie befürchtet, sie könnte in diesen Ländern gerichtlich belangt werden, wenn sie einem Kunden ein bestelltes Produkt nicht wie vereinbart liefert.\\
Weshalb kann es für einen Konsumenten in diesen Ländern von Vorteil sein, eine allfällige Rechtsstreitigkeit mit der Firma in seinem Wohnsitzland auszutragen?
\begin{itemize}
	\item Bei einer \textbf{gerichtlichen Streitigkeit} am zuständigen Gericht im Wohnsitzland des Konsumenten lassen sich die \textbf{Rechtsverfolgungskosten durch diesen eher abschätzen}, als wenn die Streitigkeit ausserhalb seines Wohnsitzlandes stattfinden würde.
	\item Die vom zuständigen Gericht am Wohnsitz des Konsumenten verwendete \textbf{Verfahrenssprache} wird dem Konsumenten \textbf{eher geläufig} sein, als im Ausland.
	\item Der Konsument ist mit dem \textbf{Rechtsystem an seinem Wohnsitz eher vertraut}, als mit einem anderen Rechtsystem.
\end{itemize}

\subsection{Straftatbestände}
Eine Bande bestellt mit Gastkonten in Online-Shops Artikel, die sie an Adressen liefern lässt, an denen sich Briefkästen mit öffentlich zugänglichen Milchkästen befinden. Laufboten holen nach der Auslieferung diese Pakete an den entsprechenden Orten ab und übergeben diese der Bande. Die Polizei kommt dahinter und erwischt einen Laufburschen auf frischer Tat. Bei der Befragung kommt heraus dass hinter dem Ganzen eine Bande aus Rumänien steckt. Die Polizei vermutet, dass die Täter den Kontakt zum Laufboten im Darknet hergestellt und so beauftragt haben.\\
Wurden ein oder mehrere Computerdelikte begangen?
\begin{itemize}
	\item Es wurde \textbf{keines der genannten Computerdelikte begangen}.
	\item Zur Auswahl stand:
	\begin{itemize}
		\item Unbefugtes Eindringen in ein Datenverarbeitungssystem (Art. 143bis StGB)
		\item Unbefugte Datenbeschaffung (Art. 143 StGB)
		\item Datenbeschädigung (Art. 144bis Ziff. 2 StGB)
	\end{itemize}
\end{itemize}