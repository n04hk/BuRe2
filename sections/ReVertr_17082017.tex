\section{Prüfung 17.08.2017}

\subsection{Kennzeichen/Wortmarke}
Die All Security AG bewirbt seit ihrer Gründung am 15. Mai 1999 ihre Dienstleistungen im Sicherheitsbereich, nämlich die Bewachung von Objekten und Personen, unter dem Kennzeichen „All Security“. Ihre Mitbewerberin, die Security for all AG, ist seit dem 1. Februar 2016 Inhaberin der Schweizer Wortmarke ALL SECURITY (Hinterlegung am 1. Oktober 2015) mit Schutzbereich „Sicherheitsdienstleistungen“ der Waren- und Dienstleistungsklasse 45 und „Versicherungsdienstleistungen“ der Waren- und Dienstleistungsklasse 35. Sind die nachstehenden Behauptungen richtig?
\begin{itemize}
	\item Die Security for all AG darf der All Security AG die \textbf{Benutzung des Kennzeichens} „All Security“ ab dem 1. Oktober 2015 \textbf{nicht} umfassend verbieten.
	\item Die Security for all AG darf der All Security AG die \textbf{Benutzung des Kennzeichens} „All Security“ ab dem 1. Februar 2016 \textbf{nicht} umfassend verbieten.
	\item All Security AG darf das Kennzeichen „All Security“ für die Bewachung von Objekten und Personen \textbf{weiterhin gebrauchen}.
	\item All Security AG darf das Kennzeichen „All Security“ für das seit dem 1. Januar 2016 von ihr neu angebotene Versicherungsprodukt für Vandalenschäden an Gebäuden \textbf{nicht} verwenden.
	\item All Security AG darf das Kennzeichen „All Security“ für das seit dem 15. Februar 2016 von ihr neu angebotene Versicherungsprodukt für Vandalenschäden an Gebäuden \textbf{nicht} verwenden.
\end{itemize}

\subsection{Urheberrecht}
\begin{itemize}
	\item Das Urheberrechts\textbf{persönlichkeits}recht ist \textbf{nicht} vererbbar.
	\item X kann sein Recht an der Namensnennung des von ihm geschaffenen Computerprogramms \textbf{nicht} auf seinen Arbeitgeber übertragen.
	\item Arbeitnehmer, der während Arbeit ein Programm geschrieben hat, darf dies \textbf{nicht} unter die Opensource-Lizenz GLP lizenzieren.
	\item Die Bearbeitung des Quellcodes ist \textbf{nicht} jedem Lizenznehmer gestattet.
	\item Eine im Handelsregister eingetragene Firma verliert ihren Rechtsschutz unter Vorbehalt der Erneuerung nach Ablauf von zehn Jahren \textbf{nicht}.
	\item Mit dem rechtmässigen Erwerb eines Programms auf DVD (in CH) darf dieses später an einen Dritten ohne Zustimmung des Rechtsinhabers in der Schweiz veräussert werden.
	\item Domain-Namen können Rechte an einer Bildmarke \textbf{nicht} verletzen.
	\item Durch Lizenzvertrag erwirbt Lizenznehmer ein Nutzungsrecht an vertragsgegenständlichen Schutzrechten mit relativer Rechtswirkung.
	\item Die rechtmässige Werkverwendung für Unterrichtszwecke (Art. 19 Abs. 1 lit. b. URG) ist entgeltlich.
	\item Datenbanken sind in EU \textbf{nicht} ausschliesslich durch Urheberrecht geschützt.
	\item Das CH-Urheberrecht erlaubt \textbf{nicht} das beliebige Kopieren von Computerprogrammen zu Privatzwecken.
	\item Der urheberrechtliche Schutz der Software stützt sich ausschliesslich auf Art. 2 Abs. 3 URG.
	\item Eine im Handelsregister des Kt.SG eingetragene Firma einer AG kann sich im Kt.GE auf ihr Ausschliesslichkeitsrecht an Firma berufen.
	\item Das Urhebernutzungsrecht ist \textbf{kein} subjektives Recht.
\end{itemize}

\subsection{Umgang mit Kundendaten}
\begin{itemize}
	\item Online-Apotheke muss Gesundheitsdaten der Kunden auf betriebseigenem Server \textbf{nicht} beim Eidgenössischen Datenschutzbeauftragten (EDÖB) anmelden, wenn sie als Inhaberin der Datensammlung einen betrieblichen Datenschutzverantwortlichen bezeichnet hat, der unabhängig die betriebsinterne Einhaltung der Datenschutzvorschriften überwacht und eine Verzeichnis der Datensammlung führt.
\end{itemize}

\subsection{Kundendaten-Weitergabe}
\begin{itemize}
	\item Firma aus Aufg.3 darf die Informationen ihrer Kundendatenbank nur mit der ausdrücklichen (schriftlichen) Einwilligung der betroffenen Kunden an Grossverteiler weitergeben.
\end{itemize}

\subsection{AGB}
\begin{itemize}
	\item Die Regelung, wonach die Sanexpress AG dem Kunden anstelle eines von diesem bestellten, aber von ihr trotz Versandbestätigung nicht lieferbares rezeptpflichtiges Medikament ein anderes Medikament zum selben Preis liefern kann, ist \textbf{ungewöhnlich und daher nicht rechtswirksam}.
	\item Bei einer Streitigkeit aus einem Kaufvertrag für Medikamente, die ein Kunde für den persönlichen Gebrauch bestellt hat, kann dieser \textbf{trotz des in den AGB vereinbarten Gerichtsstandes in Kreuzlingen beim zuständigen Gericht an seinem Wohnsitz} gegen die Sanexpress AG \textbf{klagen}.
\end{itemize}

\subsection{Straftatbestände}
Unbekannte Täter verschafften sich Zugriff auf Geschäftsmail und sandten Zahlungsaufträge an Bank der Sanexpress AG.
\begin{itemize}
	\item Straftatbestände:
	\begin{itemize}
		\item Betrügerischer Missbrauch einer Datenverarbeitungsanlage (Art. 147 StGB)
		\item Unbefugtes Eindringen in ein Datenverarbeitungssystem (Art. 143bis StGB)
	\end{itemize}
\end{itemize}

\subsection{Newsletter}
\begin{itemize}
	\item Sie hat die Kunden im Newsletter auf eine \textbf{problem- und kostenlose Ablehnung} weiterer Newsletter-Zusendungen hingewiesen.
	\item Die angeschriebenen Kunden haben der Zustellung des Newsletters zuvor \textbf{freiwillig zugestimmt}.
	\item Die Firma gibt sich unter der Angabe der \textbf{vollständigen Adresse} als Versender des Newsletters zu erkennen.
\end{itemize}

\subsection{Softwareentwicklung Lizenzen/Nutzungsrecht}
Studierende A, B und C haben in Freizeit eine App geschrieben. Keine spezielle Form der Zusammenarbeit vereinbart (A Code, B\&C UI-Design). B hat Icons aus Microsoft Word 6.0 (1993) extrahiert und verwendet. C hat für Darstellung eine proprietäre Entwickler-Lizenz einer Lizenz erworben und entnimmt relevante Teile aus Sourcecode der Bibliothek. Zwei Veröffentlichungen unter Opensource GLP3-Lizenz und kostenpflichtige Version.
\begin{itemize}
	\item A will ''seinen'' Teil (Quellcode) extrahieren und einer Firma verkaufen. Hierfür braucht er Zustimmung von B\&C!
	\item Entwicklungsprogramm von A darf nicht automatisch Rechte an mit ihr geschriebener Software erheben.
	\item Einbau von Microsoft Word 6.0 Icons stellt ein urheberrechtliches Problem dar, auch wenn es nur Bilder sind.
	\item Einbau von Microsoft Word 6.0 Icons stellt ein urheberrechtliches Problem dar, da urheberrechtlicher Schutz nicht bereits 2013 abgelaufen.
	\item Einbau und Verbreitung von Teilen des Bibliothek-Sourcecodes (C) könnte Verstoss gegen Entwickler-Lizenz darstellen.
\end{itemize}