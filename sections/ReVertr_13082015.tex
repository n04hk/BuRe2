\section{Prüfung 13.08.2015}

\subsection{Immaterialgüterrecht}
Schutzgut Ja/Nein?\\
\begin{tabular}{|l|l|}
	\hline 
	\textbf{Ja} & \textbf{Nein} \\ 
	\hline 
	Flaschenform mit individuellem Charakter & Parkleitsystem \\ 
	\hline 
	Phantasiebezeichnung einer AG & Pressefotografie von 2 Elefanten vor Kinderzoo \\
	\hline
	Logo von Unternehmen & rein funktionsbedingte Form von Konservenbüchse \\
	\hline
	Rezept für Pizzateig & Urteil des CH-Bundesgerichts \\
	\hline
	neue Arbeitstechnik für Autos & Bezeichnung ''Braun'' für Produkte im Holzverarbeitungsbereich \\
	\hline
\end{tabular}

\subsection{Urheberrecht}
\begin{itemize}
	\item Mit dem Tod des Urhebers gehen seine \textbf{Urhebernutzungsrechte} auf Erben über.
	\item Die Rechte an einem zuhause in Freizeit entwickelten PC-Programms liegen bei Firma, wenn eine \textbf{arbeitsvertragliche Regelung} zwischen Person und Firma besteht.
	\item Inhaber eines \textbf{Urheberpersönlichkeitsrecht} ist stets der Urheber.
	\item Auf Ausübung des \textbf{Urheberpersönlichkeitsrecht} kann vertraglich verzichtet werden.
	\item Benutzerhandbuch zu PC-Programm ist bei \textbf{rechtsgenüglicher Individualität} als Bestandteil des PC-Programms im Sinne von Art. 2 Abs. 3 URG \textbf{nicht} geschützt.
	\item Mit rechtmässigem \textbf{Download} eines Programms in CH darf dieses vom Erwerber später \textbf{nicht} an Dritte ohne Zustimmung des Rechtinhabers veräussert werden.
	\item Domain-Namen können Rechte an \textbf{Wortmarke} verletzen.
	\item Durch Lizenzvertrag erwirbt Lizenznehmer \textbf{nicht} ein Nutzungsrecht an vertragsgegenständlichen Schutzrechten mit absoluter Rechtswirkung.
	\item Rechtmässige Werkverwendung innerhalb des privaten Kreises aufgrund \textbf{Eigengebrauchs} (Art. 19 f. URG) ist mittelbar entgeltlich.
	\item \textbf{Datenbanken} können Werke im Sinn des Urheberrechts sein.
	\item CH-Urheberrecht erlaubt \textbf{nicht} beliebiges \textbf{Vermieten} von PC-Programmen zu Privatzwecken.
	\item Erwerber eine \textbf{Musik-CD} darf diese einem Dritten verkaufen, auch wenn Nutzungsbedingungen ein entsprechendes \textbf{Verbot} beinhalten.
	\item Eine im Handelsregister des Kt.SG eingetragene Gesellschaft kann sich in \textbf{Liechtenstein} \textbf{nicht} auf rechtlichen Schutz ihrer Firma gestützt auf die liechtensteinische Rechtsordnung berufen.
\end{itemize}

\subsection{Patentrecht}
\begin{itemize}
	\item Traditionelle Patente nach europäischem Patentrecht beinhalten Schutz von \textbf{technischen Erfindungen}.
	\item Es ist \textbf{nicht} erlaubt auf Produkte Hinweise bezüglich Patente wie folgt zu schreiben: ''\textbf{weltweit} geschützt; weltweit patentiert, weltumfassender Schutz''
	\item Eine Erfindung kann zum \textbf{Patent angemeldet} werden, wenn der Gegenstand der Erfindung neu ist, wenn der Gegenstand sich für den Fachmann nicht in nahe liegender Weise aus dem Stand der Technik ergibt und wenn die Erfindung gewerblich anwendbar ist.
	\item Man darf \textbf{Pflanzensorten oder Tierarten} sowie biologische Verfahren zur Züchtung von Pflanzen oder Tieren \textbf{nicht} zum Patent anmelden.
	\item Es ist \textbf{nicht} möglich, nur \textbf{Software}, die als ''Ansammlung abstrakter Konzepte'' verstanden wird, durch ein \textbf{Patent} in Europa und der Schweiz zu schützen.
	\item \textbf{Software} ist in Europa und der CH grundsätzlich \textbf{nicht} gestützt auf \textbf{Art. 52} EPÜ schützbar (patentfähige Gegenstände, wie Pläne, Regeln, Verfahren für gedankliche Tätigkeiten sowie Programme für Datenverarbeitungsanlagen).
	\item \textbf{Software} ist nach dem \textbf{Europäischen Patentübereinkommen} patentierbar, wenn sie Bestandteil einer Erfindung ist, welches ein technisches Problem löst (z.B. ein neuer erfinderischer Algorithmus).
\end{itemize}

\subsection{Umgang mit Kundendaten}
Firma mit Online-Jobportal. Angaben bei Bewerbung: Vor-/Nachname, Adresse, Telefonnummer, E-Mail, Geburtsdatum/-ort, Nationalität, Heimatort, Zivilstand, Aus-/Weiterbildungen, bisherige/aktuelle Arbeitsstellen sowie Referenzen. Muss diese Datensammlung beim EDÖB gemäss Art. 11a des DSG angemeldet werden?
\begin{itemize}
	\item Nein, nicht wenn die Best Sports AG als Inhaberin der Datensammlung aufgrund eines oder mehrerer im DSG und/oder in der zugehörigen Verordnung (VDSG) aufgeführten Gründen die Datensammlung nicht beim EDÖB anmelden muss (bspw. wenn die Firma einen Datenschutzverantwortlichen bezeichnet hat, der unabhängig die betriebsinterne Einhaltung der Datenschutzvorschriften überwacht und ein Verzeichnis der Datensammlungen führt).
\end{itemize}

\subsection{Weitergabe von Kundendaten}
Firma hat Jobportal, ist Tochtergesellschaft eines Konzerns mit Sitz in USA. Wann dürfen diese Daten weitergegeben werden?
\begin{itemize}
	\item Wenn die Firma mit der Muttergesellschaft in USA einen \textbf{Vertrag} abgeschlossen hat, der einen \textbf{angemessenen Datenschutz} durch Muttergesellschaft gewährleistet.
	\item Wenn die betroffenen Stellenbewerber in die Übermittlung an die Muttergesellschaft in USA \textbf{explizit eingewilligt} haben.
	\item Wenn die Muttergesellschaft beim \textbf{Handelsministerium} der USA gemäss ''US-Swiss Safer Harboer Framework'' zertifiziert ist.
\end{itemize}

\subsection{AGB}
\begin{itemize}
	\item Die \textbf{Regelung} in den AGB, wonach ein \textbf{Vertrag} zwischen Besteller und Firma zustande kommt, wenn die Firma die Bestellung unter Angabe der erwarteten Lieferzeit per E-Mail explizit bestätigt, entspricht der vorherrschenden Rechtsauffassung in CH.
	\item Die \textbf{Regelung}, wonach die Firma dem Kunden anstatt des von ihm bestellten und der Firma per E-Mail als lieferbar bestätigtes Produktes ein vergleichbares Produkt des gleichen oder eines anderen Herstellers zu einem vergleichbaren Preis zustellen kann, ist \textbf{ungewöhnlich} und daher \textbf{nicht rechtswirksam}.
\end{itemize}

\subsection{AGB}
Italiener bestellt Artikel, bekommt Bestätigung per E-Mail. In AGB steht nichts zu Lieferadressen.
\begin{itemize}
	\item Obwohl der Kunde seinen Wohnsitz in Italien hat, ist zwischen ihm und der Firma ein \textbf{gültiger Vertrag} zustande gekommen und die Firma brauch ihn zu beliefern.
	\item Falls die Firma die Artikel \textbf{mit Rechnung} zustellt, der Kunde diese nicht bezahlt	, kann die Firma \textbf{keine} \textbf{Betreibung} über den Rechnungsbetrag gegen ihn in der CH einleiten.
	\item Als Konsument mit Wohnsitz in einem \textbf{EU-Land} verfügt der Kunde über ein gesetzliches \textbf{Widerrufsrecht}.
	\item Falls die Firma nicht liefert und Der Besteller auf Lieferung besteht, ist in einem allfälligen Gerichtsverfahren damit zu rechnen, dass das zuständige Gericht die Streitigkeit \textbf{nach italienischem Recht} beurteilen wird.
\end{itemize}

\subsection{Straftatbestände}
Täter drangen in Netzwerk der Bank ein und transferierten Gelder von Kundenkonten auf eigene im Ausland. Geschafft mit Phishing-E-Mails an Mitarbeiter der Bank. Durch klicken eines Links gelang der Zugriff.
\begin{itemize}
	\item \textbf{Unbefugtes Eindringen} in ein Datenverarbeitungssystem (Art. 143bis StGB)
	\item \textbf{Datenbeschädigung} (Art. 144bis Ziff. 2 StGB)
	\item \textbf{Betrügerischer Missbrauch} einer Datenverarbeitungsanlage (Art. 147 StGB)
\end{itemize}

\subsection{Internetverkehr auswerten}
Möglicher Täter von vorheriger Aufgabe gefunden, Internetverkehr soll nachträglich auswerten.
\begin{itemize}
	\item Die Herausgabe der Randdaten über den Internetverkehr auf Laptop des Verdächtigen muss von Staatsanwaltschaft angeordnet und vom zuständigen Zwangsmassnahmengericht bewilligt werden.
	\item Der CH Internetanbieter des Verdächtigen muss die Randdaten über dessen Internetverkehr \textbf{nicht} gemäss Gesetz während mind. 18 Monaten aufbewahren.
	\item Der CH Internetanbieter des Verdächtigen muss einer Herausgabe der Rohdaten bei Vorliegen einer gültigen Bewilligung des Zuständigen Zwangsmassnahmengerichts \textbf{nicht} explizit zustimmen.
	\item Die Auswertung von Randdaten über den Internetverkehr von Tatverdächtigen ist gemäss Ch Gesetzgebung für Computerdelikte \textbf{nicht} von vornherein ausgeschlossen.
\end{itemize}