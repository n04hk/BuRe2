\section{Prüfung 14.08.2013}

\subsection{Immaterialgüterrecht}
Rechts- oder Schutzgut Ja/Nein?\\
\begin{tabular}{|l|l|}
	\hline 
	\textbf{Ja} & \textbf{Nein} \\ 
	\hline 
	Halbleiterchip & Bezeichnung ''0'' für Zahnpasta \\ 
	\hline 
	Bild von Pablo Picasso & Gesetzesbestimmung aus ZGB \\
	\hline
	Werbe-Jingle von Migros & \\
	\hline
	Pizzateig-Rezept &  \\
	\hline
	Verfahren zum Trennen von Gemischen mit Salzlösung & \\
	\hline
	Grundrissplan eines schützenswerten Gebäudes &  \\
	\hline
\end{tabular}

\subsection{Urheberrecht}
\begin{itemize}
	\item Mit dem \textbf{Tod} des Urhebers geht auch sein \textbf{Urheberpersönlichkeitsrecht} unter.
	\item Das in Art. 12 Abs. 2 URG (Erschöpfungsgrundsatz) stipulierte Recht zum Gebrauch und zur Weiterveräusserung eines \textbf{Computerprogramms} ist \textbf{keine} vertraglich begründete \textbf{Einzellizenz}.
	\item Das in Art. 12 Abs. 2 URG (Erschöpfungsgrundsatz) stipulierte Recht zum Gebrauch und zur Weiterveräusserung eines \textbf{Computerprogramms} darf zwischen dem \textbf{Rechteinhaber} und dem \textbf{Nutzer} vertraglich eingeschränkt werden.
	\item Art. 12 Abs. 2 URG (Erschöpfungsgrundsatz) ist \textbf{nicht} beim \textbf{Download} eines Computerprogramms anwendbar.
	\item Das Recht auf die \textbf{Marke} entsteht bei einer Eintragung der Marke rückwirkend ab ihrer \textbf{Hinterlegung}.
	\item Ein \textbf{Domain-Name} ist \textbf{keine} Marke im Sinne des Markenschutzgesetzes.
	\item \textbf{Domain-Namen} können die Rechte an einer Firma \textbf{verletzen}.
	\item Der \textbf{Switch-Domain-Namen-Streitbeilegungsdienst} ist \textbf{kein} staatliches Gericht.
	\item Auf einen \textbf{Lizenzvertrag} kommen \textbf{nicht} hauptsächlich die Bestimmungen des \textbf{Auftragsrechts} (Art. 394 ff. OR) analog zur Anwendung.
	\item Das Recht veröffentlichte Werke zum \textbf{Privatgebrauch} zu nutzen (Art. 19 Abs. 1 lit. a URG) ist \textbf{entgeltlich}.
	\item Das \textbf{CH-Urheberrecht} erlaubt das Kopieren von \textbf{PC-Programmen} zu \textbf{Unterrichtszwecken} \textbf{nicht}.
	\item Der Erwerber einer \textbf{Musik-CD} darf diese seiner Mutter \textbf{vermieten}.
	\item Ein ausländischer Rechteinhaber mit \textbf{Wohnsitz in CH} kann sich hinsichtlich seiner \textbf{Schutzrechte} an \textbf{Immaterialgütern} innerhalb der CH auf das \textbf{CH-Immaterialgüterrecht} berufen.
	\item Ein ausländischer Rechteinhaber mit \textbf{Wohnsitz im Ausland} kann sich hinsichtlich seiner \textbf{Schutzrechte} an \textbf{Immaterialgütern} innerhalb der CH auf das \textbf{CH-Immaterialgüterrecht} berufen.
	\item Ein CH-Rechteinhaber mit \textbf{Wohnsitz in CH} kann sich hinsichtlich seiner \textbf{Schutzrechte} an \textbf{Immaterialgütern} auf ausländischem Staatsgebiet \textbf{nicht} auf das \textbf{CH-Immaterialgüterrecht} berufen.
	\item \textbf{Datenbanken} sind in den Mitgliedsstaaten der \textbf{EU} vordergründig \textbf{nicht} durch das Urheberrecht geschützt.
	\item Der \textbf{Lizenzvertrag} ist \textbf{kein} im Gesetz ausdrücklich geregelter Vertrag.
	\item Im Falle einer \textbf{ausschliesslichen Softwarelizenz} ist es dem Lizenzgeber untersagt, die lizenziert Software zu nutzen.
\end{itemize}

\subsection{Firmenschutz}
Pflegehilfe Schweiz AG erbringt Dienstleistungen im Bereich der Haushaltshilfe und Pflege, einschliesslich Personalvermittlung und Personalverleih.
\begin{itemize}
	\item ''Pflegehilfe Schweiz AG'' ist \textbf{kein} Kennzeichen, welches eine Ware oder eine Dienstleistung eines Unternehmens beschreibt.
	\item ''Pflegehilfe Schweiz AG'' ist \textbf{nicht} im CH-Firmenregister eingetragen.
	\item Der \textbf{Firmenschutz} besteht solange als das betreffende Unternehmen besteht.
	\item Pflegehilfe Schweiz AG ist ein Unternehmen mit Sitz im \textbf{Kt. ZG}. ''Pflegehilfe Schweiz AG'' darf daher \textbf{nicht} von einem anderen Unternehmen mit Sitz im \textbf{Kt. ZH} als Firma verwendet werden.
\end{itemize}

\subsection{Markenrecht}
Pflegehilfe Schweiz AG möchte gern ''Pflegehilfe Schweiz'' als Wortmarke für die Warenklasse 44 ''Ärztliche Versorgung'' ''Krankenpflegedienste'', ''Gesundheitspflege'' im CH-Markenregister eintragen lassen.
\begin{itemize}
	\item Die Eintragung wird vom Institut für Geistiges Eigentum \textbf{verweigert}.
\end{itemize}

\subsection{AGB/Privacy Policy}
Welche Fragen sollte ein Kunde aufgrund der Privacy Policy der Firma beantworten können?
\begin{itemize}
	\item Welche \textbf{Daten} zu welchen Zwecken erhoben werden.
	\item Ob und welche \textbf{Sicherheitsmassnahmen} zum Datenschutz angewendet werden.
	\item Ob und an welche namentlich genannte Dritte Daten zu welchen Zwecken \textbf{weitergegeben} werden.
\end{itemize}
\begin{Large}
Folgende \textbf{nicht}:
\end{Large}
\begin{itemize}
	\item Den Gerichtsstand und das anwendbare Recht für den Fall von allfälligen Streitigkeiten über bestellte Produkte zwischen Kunden und der Firma.
	\item Die Zahlungs- und Lieferbedingungen von Produkten im Online-Shop.
	\item Die Regelung von Garantie und Haftung für Produkte.
\end{itemize}

\subsection{Kundendaten}
Kundenbindungsprogramm (Customer Relationship Management), welche Grundsätze treffen zu?
\begin{itemize}
	\item Es dürfen nur jene Kundendaten \textbf{erhoben} werden, welche für die Verwendung innerhalb des Kundenbindungsprogramms \textbf{geeignet} und \textbf{erforderlich} sind.
\end{itemize}
\begin{Large}
Folgende \textbf{nicht}:
\end{Large}
\begin{itemize}
	\item Im Rahmen eines Kundenbindungsprogramms dürfen nur der \textbf{Vor- und Nachname} sowie die \textbf{Adresse} eines Kunden, jedoch \textbf{keine weiteren Daten} erhoben werden.
	\item Die Zustimmung des Kunden zur Verwendung seiner Daten im Rahmen eines Kundenbindungsprogramms erlaubt gleichzeitig die \textbf{Weitergabe von Kundendaten} an Dritte.
	\item Kundendaten dürfen im Rahmen eines Kundenbindungsprogramms \textbf{unter keinen Umständen} für \textbf{Marketingzwecke} verwendet werden.
\end{itemize}

\subsection{AGB/Kaufvertrag}
Wer stellt den Antrag auf Abschluss des Kaufvertrags?
\begin{itemize}
	\item Bestellung eines Produkts im \textbf{Online-Shop}
	\begin{itemize}
		\item Kunde
	\end{itemize}
	\item Versteigerung im \textbf{Internet-Auktionsportal} (bspw. ebay)
	\begin{itemize}
		\item Firma
	\end{itemize}
	\item Auslage von Ware im \textbf{Ladenlokal}
	\begin{itemize}
		\item Firma
	\end{itemize}
\end{itemize}

\subsection{Digitale Signatur}
Welcher Form entspricht die digitale Signatur aus rechtlicher Sicht?
\begin{itemize}
	\item Der einfachen Schriftlichkeit
	\item \textbf{Nicht} der öffentlichen Beurkundung
	\item \textbf{Nicht} der qualifizierten Schriftlichkeit
	\item \textbf{Nicht} gar keiner Form
\end{itemize}

\subsection{Straftatbestände}
E-Banking-Login mit SMS, Täter senden bei Login-Versuch eine Fehlermeldung anstatt Login-Code per SMS mit der Aufforderung, über einen Link ein neues Sicherheitszertifikat (in Wahrheit ein Virus) zu installieren. Nach Installation wird zwar bei nächstem Login-Versuch des Nutzers ein SMS erhalten, jedoch mit ungültigem Code, also kann er sich nicht einloggen. Täter können sich mit dem richtigen Code einloggen und Konto leerräumen.\\
Wurde ein Straftatbestand erfüllt, wenn ja, welcher?
\begin{itemize}
	\item Ja, \textbf{Betrügerischer Missbrauch einer Datenverarbeitungsanlage} (Art. 147 StGB)
\end{itemize}