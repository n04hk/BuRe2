\section{Prüfung 12.08.2012}
\subsection{Immaterialgüterrecht}
Rechts- oder Schutzgut Ja/Nein?\\
\begin{tabular}{|l|l|}
	\hline 
	\textbf{Ja} & \textbf{Nein} \\ 
	\hline 
	Drei-Dimensionale Form der ''Toblerone'' & Evangelium nach Markus (Bibel, Neues Testament) \\ 
	\hline 
	Bezeichnung ''Braun'' für Nahrungsergänzungsmittel & Fotoschnappschuss, Familie auf Titlis \\
	\hline
	Vereinsname ''Stadtsänger Rapperswil'' & Kundenlisten eines Unternehmens \\
	\hline
	Computerprogramm & Frisch-Pizza-Verkaufs-Know-How eines Unternehmens \\
	\hline
\end{tabular}

\subsection{Urheberrecht}
\begin{itemize}
	\item Urheber eines Werks ist stets eine \textbf{natürliche Person}.
	\item Der \textbf{Lizenzvertrag} räumt den Lizenznehmern eine \textbf{absolute Rechtsposition} am Lizenzgut ein.
	\item Der \textbf{Markenschutz} hält \textbf{nicht} höchstens zehn Jahre nach seiner \textbf{Hinterlegung} an.
	\item Der Eintrag einer neuen Firma einer AG, welche \textbf{ähnlich} ist, mit einer bereits im \textbf{Handelsregister} bestehenden Firma, wird im Handelsregister eingetragen.
	\item Mit dem Eintrag einer \textbf{CH-Firma} ist diese \textbf{nicht} \textbf{weltweit} geschützt.
	\item Rapperswil-Jona AG würde als Firma im \textbf{Handelsregister} eingetragen werden.
	\item Das Recht auf \textbf{Anerkennung der Urheberschaft} (Art. 9 Abs. 1 URG) ist gemäss CH-Rechtsordnung \textbf{weder übertragbar noch vererblich}.
	\item Mit der \textbf{Löschung} der Firma im Handelsregister endet auch der \textbf{Firmenschutz}.
	\item Auf einen \textbf{Entwicklungsvertrag} kommen die \textbf{Bestimmungen des Auftragsrechts} (Art. 394 ff. OR) zur Anwendung.
	\item{Verwertungsgesellschaften} nehmen die \textbf{Rechte der Urheber} wahr.
	\item Der Erwerber eines \textbf{Computerprogramms} darf dieses einem \textbf{Dritten} weiter \textbf{veräussern}.
	\item Der Erwerber eines \textbf{Computerprogramms} darf dieses seiner Mutter \textbf{nicht} \textbf{vermieten}.
	\item Ein \textbf{Domain-Name} ist \textbf{keine} \textbf{Marke}.
	\item Die Verwendung eines \textbf{Domain-Namens} kann gegebenenfalls die Rechte eines \textbf{Markeninhabers} verletzen.
	\item Ein \textbf{Domain-Name} ist ein \textbf{Kennzeichen}.
	\item Der Plan eines \textbf{Architekten} für ein Einfamilienhaus ist \textbf{nicht} durch das \textbf{Designrecht} geschützt.
	\item Der \textbf{Lizenzvertrag} ist \textbf{nicht} im \textbf{CH-Obligationenrecht} geregelt.
\end{itemize} 

\subsection{Schutzrecht: Software-Unternehmen}
Vereinbarung mit einem Kunden: Die \textbf{Schutzrechte} an der Software stehen der Muster AG \textbf{ausschliesslich} zu. Diese räumt aber dem Kunden daran ein \textbf{persönliches} sowie \textbf{nicht ausschliessliches} jedoch \textbf{örtlich, sachlich und zeitlich unbeschränktes Nutzungsrecht} ein, welches zudem das \textbf{Recht zur Bearbeitung} und damit die Nutzung der Software als Grundlage eines \textbf{neuen schutzfähigen Arbeitsergebnisses} beinhaltet. Die Muster AG stellte hierfür dem Kunden den Quellcode, die Programmbeschreibungen sowie Dokumentationen nach vollständiger Bezahlung der Vergütung in geeigneter Form zur Verfügung.
\begin{itemize}
	\item Der Kunde darf die Software einer \textbf{anderen Firma übertragen}, \textbf{ohne} sie \textbf{selber} künftig \textbf{zu nutzen}.
	\item Der Kunde darf die Software in \textbf{beliebiger Anzahl kopieren}.
	\item Der Kunde darf die Software \textbf{verändern}.
	\item Der Kunde darf die Software \textbf{nicht} an eine andere Firma übertragen und dieser das Recht einräumen, davon \textbf{beliebig viele Kopien} zu machen.
	\item Der Kunde darf die Software \textbf{nicht} an eine andere Firma übertragen und dieser zudem das Recht einräumen, diese zu \textbf{verändern}.
	\item Der Kunde darf die gestützt auf die von der Muster AG erworbene Software erstellten und darauf basierenden \textbf{eigenen Applikationen} \textbf{nicht} an eine weitere Firma übertragen und gleichzeitig diese \textbf{weiterhin in ihrem Betrieb verwenden}.
	\item Der Kunde darf die Software \textbf{in den USA verändern}.
	\item Die \textbf{Tochtergesellschaft} des Kunden darf die Software \textbf{nicht} \textbf{gleichzeitig} und \textbf{neben} dem Kunden beliebig bestimmungsgemäss verwenden. (Laden, Anzeigen, Ablaufen, Speichern)
\end{itemize}

\subsection{Umgang mit Kundendaten}
X nimmt an einer Verlosung teil. Er gewinnt nichts, bekommt aber nach der Teilnahme laufend Werbung und andere Angebote per Post.\\
Darf die Firma die infolge der Teilnahme am Wettbewerb erhaltene Adresse zur Zustellung von Werbung verwenden?
\begin{itemize}
	\item Ja, wenn X die Firma zuvor unabhängig von der Teilnahme am Wettbewerb ermächtigt hat, seine Adresse auch zu Werbezwecken verwenden zu dürfen.
\end{itemize}
\begin{Large}
	\textbf{Falsch}:
\end{Large}
\begin{itemize}
	\item Ja, denn eine Teilnahme an einem Wettbewerb schliesst automatisch die Zustimmung des Teilnehmers zur Zustellung von Werbung mit ein.
	\item Nein, denn die Firma hätte X nur Werbung zustellen dürfen, wenn X bei der
	Verlosung einen Preis gewonnen hätte.
	\item Ja, wenn die Firma X zuvor nach dem Besitz eines gültigen Führerausweis gefragt und
	X die Frage mit ja beantwortet hat.
	\item Nein, denn die Firma hätte die Adresse von X zu Werbezwecken nur gestützt auf eine
	gesetzliche Grundlage verwenden dürfen.
\end{itemize}

\subsection{Umgang mit Kundendaten: Gesetz}
Nach welchem Gesetz beurteilt sich die Verwendung der Adresse von X durch die Firma?
\begin{itemize}
	\item Nach dem \textbf{Bundesgesetz über den Datenschutz}, weil es sich bei der Firma um ein \textbf{privates Unternehmen} handelt. 
\end{itemize}
\begin{Large}
	Folgende \textbf{nicht}:
\end{Large}
\begin{itemize}
	\item Nach dem Bundesgesetz über den Datenschutz, weil es sich bei X um eine Privatperson handelt.
	\item Nach dem Gesetz über die Information und den Datenschutz des Kantons Zürich, weil die Firma ihren Sitz in Zürich hat.
\end{itemize}

\subsection{AGB}
Internet-Schuhhändler mit Sitz im Kt. SG:\\
Durch das Abschicken einer Bestellung erklären Sie sich ausdrücklich mit diesen AGB einverstanden. Der Vertrag kommt zustande, wenn wir Ihnen die Bestellung der gewünschten Ware per E-Mail bestätigen. Die Lieferfrist beträgt grundsätzlich 20 Tage ab Bestätigung der Bestellung, kann jedoch nicht garantiert werden. Rechnungen müssen innert 10 Tagen nach Erhalt der Ware bezahlt werden. Jede Haftung für die gelieferten/verkauften Produkte wird ausgeschlossen. Das Rechtsverhältnis untersteht CH-Recht. Ausschliesslicher Gerichtsstand für sämtliche Streitigkeiten ist St. Gallen.\\
Könnten diese AGB rechtlich problematisch sein und wenn ja, weshalb?
\begin{itemize}
	\item Der \textbf{Gerichtsstand am Sitz der Firma in St. Gallen ist ungültig}, da die Bestellung von Artikeln durch eine Privatperson als Konsumentenvertrag zu qualifizieren ist und ein Konsument mit Wohnsitz in der Schweiz nicht zum Voraus gültig auf den Gerichtsstand an seinem Wohnsitz verzichten kann.
\end{itemize}
\begin{Large}
	Folgende \textbf{nicht}:
\end{Large}
\begin{itemize}
	\item In CH wohnhaften Konsumenten muss zwingend ein Widerrufsrecht von 14 Tagen seit Vertragsabschluss gewährt werden.
	\item Der Hinweis, dass der Vertrag zustande kommt, wenn die Bestellung vom Anbieter per E-Mail bestätigt wird, ist falsch. Die Angebote des Anbieters im Internet stellen einen verbindlichen Antrag desselben dar, womit ein Vertrag bereits zustande kommt, wenn ein Kunde die Bestellung für ein Produkt abgibt und damit den Antrag des Anbieters annimmt.
	\item Die AGB der Firma sind rechtlich unproblematisch.
\end{itemize}

\subsection{Verkauf ins Ausland}
Die Firma aus vorheriger Aufgabe möchte auch Kunden mit Wohnsitz in Italien bedienen.\\
Hat die Firma aus rechtlicher Sicht etwas zu beachten?
\begin{itemize}
	\item Falls Konsumenten mit Wohnsitz in Italien ihre Rechnung für elektronisch bestellte Artikel nicht bezahlen, muss die Firma damit rechnen, dass sie solche Konsumenten in Italien	belangen muss.
	\item Konsumenten mit Wohnsitz in Italien verfügen automatisch über ein 7-tägiges Widerrufsrecht für im Internet bestellte Artikel.
	\item In einem allfälligen Gerichtsverfahren gegen einen in Italien wohnhaften Konsumenten muss die Firma damit rechnen, dass das zuständige Gericht die Streitigkeit nach italienischem Recht beurteilen wird
\end{itemize}

\subsection{Straftatbestände}
H arbeitet bei einer CH-Grossbank. Er hat R und M je ein Konto eröffnet und will diesen Gutschriften überweisen, die sie danach untereinander aufteilen. H entwendete einem Arbeitskollegen dessen Leigimationskarte, die unbeaufsichtigt auf dem Schreibtisch lag. Der Code dazu stand auf einem Zettel in einer unverschlossenen Schreibtischschublade. Er überwies beiden je etwas über 3 Mio. Fr. R versuchte erfolglos Geld von diesem Konto zu beziehen, M hob einen Betrag von 74'400 Fr. ab.\\
Hat H sich strafbar gemacht und falls ja, inwiefern?
\begin{itemize}
	\item \textbf{Betrügischer Missbrauch einer Datenverarbeitungsanlage} (Art. 147 StGB)
\end{itemize}

\subsection{Schadenersatz nach Straftat}
Die Grossbank aus vorheriger Aufgabe möchte rechtliche Schritte gegen H, R, M und P einleiten und Schadenersatz geltend machen. Gegenüber wem und gestützt auf welche rechtliche Grundlage im OR kann die Grossbank mit Aussicht auf Erfolg Schadenersatz geltend machen?
\begin{itemize}
	\item Gegenüber P gestützt auf \textbf{vertragliche Haftpflicht} (Art. 97 OR in Verbindung mit Art. 321e OR)
	\item Gegenüber R gestützt auf Art. 41 OR (\textbf{ausservertragliche Haftpflicht})
	\item Gegenüber M gestützt auf Art. 41 OR (\textbf{ausservertragliche Haftpflicht})
	\item Gegenüber H gestützt auf Art. 97 OR in Verbindung mit Art. 321e OR (\textbf{vertragliche Haftpflicht})
\end{itemize}
